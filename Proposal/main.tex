\documentclass[sigplan,screen]{acmart}
%% NOTE that a single column version is required for 
%% submission and peer review. This can be done by changing
%% the \doucmentclass[...]{acmart} in this template to 
%% \documentclass[manuscript,screen,review]{acmart}
%% 
%% To ensure 100% compatibility, please check the white list of
%% approved LaTeX packages to be used with the Master Article Template at
%% https://www.acm.org/publications/taps/whitelist-of-latex-packages 
%% before creating your document. The white list page provides 
%% information on how to submit additional LaTeX packages for 
%% review and adoption.
%% Fonts used in the template cannot be substituted; margin 
%% adjustments are not allowed.
%%
%% \BibTeX command to typeset BibTeX logo in the docs
\AtBeginDocument{%
  \providecommand\BibTeX{{%
    \normalfont B\kern-0.5em{\scshape i\kern-0.25em b}\kern-0.8em\TeX}}}

\begin{document}
\title{CS512 Project Proposal: Leveraging Document Co-Clicks To Infer Similarities}


\author{Daniel Campos} \author{Revanth Reddy}
\author{Shweta Garg} 
\email{dcampos3, revanth3, shwetag2 @illinois.edu}


%%
%% The abstract is a short summary of the work to be presented in the
%% article.
\begin{abstract}
In the last few decades search engines like Google, Bing, Baidu, and Yandex have become the primary way that people around the world interact with information. Their constant and diverse usage has made these engines ideal sources for training data like document co-clicks. Using the ORCAS dataset we seek to explore the intersection of Data Mining and Deep Learning and answer the question: Can document co-clicks be used to learn similarity between concepts?
\end{abstract}

%%
%% The code below is generated by the tool at http://dl.acm.org/ccs.cfm.
%% Please copy and paste the code instead of the example below.
%%
\begin{CCSXML}
<ccs2012>
<concept>
<concept_id>10002951.10003317.10003325.10003330</concept_id>
<concept_desc>Information systems~Query reformulation</concept_desc>
<concept_significance>500</concept_significance>
</concept>
<concept>
<concept_id>10002951.10003317.10003325.10003328</concept_id>
<concept_desc>Information systems~Query log analysis</concept_desc>
<concept_significance>500</concept_significance>
</concept>
<concept>
<concept_id>10002951.10003227.10003351.10003446</concept_id>
<concept_desc>Information systems~Data stream mining</concept_desc>
<concept_significance>500</concept_significance>
</concept>
</ccs2012>
\end{CCSXML}

\ccsdesc[500]{Information systems~Query reformulation}
\ccsdesc[500]{Information systems~Query log analysis}
\ccsdesc[500]{Information systems~Data stream mining}

%%
%% Keywords. The author(s) should pick words that accurately describe
%% the work being presented. Separate the keywords with commas.
\keywords{datasets, information retrieval, data mining, query rewriting, synonym prediction}

%%
%% This command processes the author and affiliation and title
%% information and builds the first part of the formatted document.
\maketitle
\section{Introduction}
Being able to understand the similarity between words is a core component of many modern computational systems. From a search engine, which needs to understand that the user queries \textit{How tall is tom cruise.} and \textit{tom cruise height?} have the same intent. To an e-commerce website that can use similarity between \textit{car parts} and \textit{auto parts} to provide similar product recommendations being able to recognize and understand similarity between words is crucial. Supervised methods tend to rely on carefully constructed synonym dictionaries or labeled datasets. Unsupervised methods like contextual \cite{Devlin2019BERTPO} and static word embeddings \cite{Mikolov2013DistributedRO}  provide a scalable and highly accurate notion of synonym search using vector similarity but once again rely on a large and diverse corpus for pretraining. Whether labeled or unlabeled, low-resource languages do not always have the luxury of large data sources. As a result learning methods that find alternative methods of dataset construction will prove useful. \\ \\
Modern commercial search engines have hundreds of millions of daily active users around the world. While users may speak different languages and search for different document the use of search engines as a source of knowledge brings them together. Every time a user issues a query and engages with a document they are providing a signal to the search engine. User click have long been used to improve information retrieval systems \cite{Chuklin2013ClickMI} but because of the sensitive nature of users queries few datasets have been released publicly and there has been even less public research into these datasets. \\ \\
Building on the Question answering and Information Retrieval benchmark, MSMARCO \cite{Campos2016MSMA}, the ORCAS \cite{Craswell2020ORCAS2M} dataset features the largest publicly accessible document click dataset. It features 18.4 million document URL clicks on 10.4 million queries and 1.4 million documents which were extracted from the logs of a commercial search engine. This dataset captures both the similarity and difference of user searches. For example, the query \textit{pandas} features clicks on documents relating to the python programming library and the animal yet the query \textit{panda} features clicks on documents related to the restaurant chain panda express and the animal. \\ \\
Using the ORCAS dataset, we seek to explore a broad array of data mining techniques to produce novel training data to be used for transfer learning in tasks like query rewriting and synonym prediction. We will explore the usage dynamic network alignment, topic clustering and link prediction to produce training data which we will then use to explore its effects on the aforementioned tasks. Our goal is to explore how well click based data can replace traditional training corpora. \\
\section{Main Ideas}
The main focus of our work is to explore what kind of value we can extract out of the ORCAS dataset with regards to using it as distant supervision data. We believe that if our efforts are successful this method can be explored for low resource languages.
\subsection{Dataset Mining}
The focus of our work is to start with general exploration and processing of the ORCAS dataset. We will explore the application of traditional data mining algorithms to explore what signals we can extract. We seek to create a set of signals which we can be used in our downstream experiments to prove or disprove our hypothesis. Some of the methods we plan to explore include: phrase based clustering, click based query clustering, n-gram association analysis, etc.
\subsection{Query Synonym Prediction}
Using the ORCAS dataset, we formulate the query synonym prediction task as a task where the goal is to identifying the word synonyms in queries. These synonyms can then be used to identify similar queries efficiently. //
In this work, we intend to use contextual and non contextual word representations to explore how these methods can represent n-gram and query similarity, and leverage them to build our own knowledge base of query synonyms. We plan to evaluate our approach by predicting the degree of connection between queries (measured by proximity in click graph) and between n-gram terms. Another way to evaluate our mined synonyms would be to compare them against those from English dictionaries such as Merriam-Webster dictionary. We also plan to perform a qualitative evaluation by getting a small portion of the generated query synonyms analyzed by human evaluators. 

To further explore how such user logs can help learn signals about query similarities, we plan to leverage the query co-clicks from ORCAS in a transfer learning setting for a related task of Quora duplicate question detection\footnote{\href{https://www.quora.com/q/quoradata/First-Quora-Dataset-Release-Question-Pairs}{Quora Duplicate Question Detection}}. We will investigate how well a model trained on the query co-clicks from ORCAS performs on the Quora dataset and will also explore into using the co-clicks data to come up with simpler methods that can match the performance of more complex methods \cite{chen2018quora, chandra2020experiments, abishek2019enhanced, dadashovquora}.

%Once we have produced various training datasets we will explore how these datasets can be used for transfer learning. One task in this direction is to perform Query Synonym Prediction that is defined as a task that infers word synonyms from the co-clicks in the ORCAS dataset. These synonyms can then be used to identify similar queries efficiently.



%Previously the research community has extensively worked on the publicly available Quora Dupilcate Questions Dataset\footnote{https://www.quora.com/q/quoradata/First-Quora-Dataset-Release-Question-Pairs} released by the popular Question Answering platform themselves. Several works \cite{chen2018quora, chandra2020experiments, abishek2019enhanced , dadashovquora} on this dataset have used concepts such as feature engineering (feature extraction, term frequency-inverse document frequency (TF-IDF), word embeddings etc.) with Neural Network models (LSTM, RNN, CNN etc.) to identify the duplicate questions in this dataset. However these works make use of complex models for this task. To simplify this we believe that we can use the synonyms generated by our model as features on the Quora Duplicate Questions dataset and fine tune our pre-trained model to perform the same task of detecting the duplicate questions effectively using non-complex methods.

%Another interesting work HolisticOpt \cite{he2016automatic} aligns with our thought process as it makes use of query log clicks and web table attribute name co-occurrences to find attribute synonyms that can be used to boost the performance of search engines.

%We intend to use contextual and non contextual word representations to explore how these methods can represent n-gram and query similarity and build our own dataset containing the query synonyms that are generated. The evaluation can be both quantitative and qualitative where for quantitative analysis we will focus on predicting the degree of connection between queries (measured by proximity in click graph) and between n-gram terms. We will also compare our results with baseline methods such as using the Quora Duplicate Question dataset or evaluating the quality of our synonyms using the synonyms mined from English dictionaries such as Merriam-Webster dictionary. The qualitative evaluation for our task will include getting the generated query synonyms analyzed by human evaluators to check the quality.

\subsection{Query Rewriting}
Query rewriting is the process of automatically expanding a search query to better understand the user's intent. Query rewriting is typically used for improving the recall of IR systems, to retrieve a larger set of relevant results. We propose to use queries that have the same co-clicks in ORCAS as parallel data to train a sequence-to-sequence model to do query re-writing. Specifically, we intend to finetune a pre-trained encoder-decoder model \cite{lewis2020bart, raffel2020exploring} using these similar query pairs. Then, we plan to explore how our query re-writing approach can be used to improve established ranking models like BM25 \cite{robertson2009probabilistic} and neural IR models \cite{karpukhin2020dense, khattab2020colbert} on popular IR benchmarks like MS MARCO \cite{Campos2016MSMA}.

Another potential direction to explore is converting keyword based queries to natural language queries. Current neural network based IR systems, which are built for semantic search, might fall short against keyword-based queries. Hence, converting such queries into natural language would not just improve IR performance but also help understand searcher intent and query context, which are critical in query understanding. To achieve this, we plan to create parallel data from co-clicks, with smaller queries considered to be keyword-based and those with more words categorized as natural language queries. We will evaluate our approach using the TREC Web track\footnote{\href{https://trec.nist.gov/data/webmain.html}{https://trec.nist.gov/data/webmain.html}}, which contains both keyword-based and natural language questions with information seeking behaviors that are common in web search. 

%Similar to synonym detection, we will explore the effect of using our processed dataset on information retrieval centric domain. Using the MSMARCO baseline (other baselines?) we will explore how the use of our processed data can be applied to improve existing and established ranking models like BM25.  
\section{Preliminary Plan}
Our plan essentially has two stages: data exploration and transfer learning. 
\subsection{Milestones}
\begin{enumerate}
    \item Project Scoping(Feb 25th): Provide a scope of work to be attempted and problem framing.
    \item Data Exploration and Clustering(March 15th): Application of Data Mining algorithms to visualize and explore clusters both of query terms and documents.
    \item Transfer Learning Labels(March 28): Using finding from data exploration and clustering we will create processed data to be used with our transfer tasks.
    \item Synonym Evaluation dataset and Baselines(March 28th): Finalize query synonym prediction task and produce baselines that do not leverage click data.
    \item Query Rewriting baseline(March 28): Finalize query rewriting task and produce baselines that do not leverage click data.
    \item Midterm Report(March 30): Discuss progress and learning.
    \item Experiments across tasks(April 15th): Initial results using transfer data on benchmark tasks. Use results to go back and tweak mining methods.
    \item Experiments across tasks v2(April 30): Updated results using improved data.
    \item Final report (May 5th)
\end{enumerate}
\subsection{Roles}
\begin{itemize}
  \item \textbf{Daniel Campos}: Application of Data Mining techniques to dataset for exploration and transfer label creation, report writing, IR experiments
  \item \textbf{Revanth Reddy}: Query rewriting baseline, experiments and tweaking, IR experiments.
  \item \textbf{Shweta Garg}: Query Synonym Prediction task analysis, survey on existing baselines and their implementation, experiments and tweaking. 
\end{itemize}

\section{Relevant Research}
We break our relevant reading into 4 sections: network mining, transfer learning and distant supervision, information retrieval, and synonym detection. Each section includes an initial paper list.
\subsection{Network and Click Mining}
\cite{Li2013EnhancedIR}
\cite{Cao2008ContextawareQS}
\cite{Hua2013ClickageTB}
\cite{Chuklin2013ClickMI}
\cite{Kacprzak2017AQL}
\cite{FournierViger2017ASO}
\cite{Ghosh2012ATR}
\subsection{Transfer Learning and Distant Supervision}
\cite{Mintz2009DistantSF}
\cite{Hedderich2020TransferLA}
\cite{Xu2020QueryFM}
\cite{Ji2017DistantSF}
\cite{Mitra2020NeuralMF}
\subsection{Information Retrieval and Query Rewriting}
\cite{Strohman2005IndriA}
\cite{Cheriton2019FromDT}
\cite{Jiang2016LearningQA}
\cite{Radlinski2010InferringQI}
\subsection{Query Similarity}
\cite{Ansari2020IdentifyingSD}
\cite{10.1145/3292500.3330914}
\cite{Sharma2019NaturalLU}
\cite{Viswanathan2019DetectionOD}
\cite{Bona2010LearningDM}
\bibliographystyle{ACM-Reference-Format}
\bibliography{bibliography}
%\appendix
\end{document}
\endinput

