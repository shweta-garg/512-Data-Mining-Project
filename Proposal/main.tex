\documentclass[sigplan,screen]{acmart}
\AtBeginDocument{%
  \providecommand\BibTeX{{%
    \normalfont B\kern-0.5em{\scshape i\kern-0.25em b}\kern-0.8em\TeX}}}

\begin{document}
\title{CS512 Proposal: ORCAS Migration: How can user document co-clicks be for downstream tasks}

\author{Daniel Campos} \And \author{Shweta Garg} \And \author{Revanth Reddy


%%
%% The abstract is a short summary of the work to be presented in the
%% article.
\begin{abstract}
  \textbf{Dynamic Network Alignmen}, Distant Supervision
  
  Distant supervision for relation
extraction without labeled data 
Clustype: Effective
entity recognition and typing by relation phrase-based clusteringA clear and well-documented \LaTeX\ document is presented as an
  article formatted for publication by ACM in a conference proceedings
  or journal publication. Based on the ``acmart'' document class, this
  article presents and explains many of the common variations, as well
  as many of the formatting elements an author may use in the
  preparation of the documentation of their work.
\end{abstract}

%%
%% The code below is generated by the tool at http://dl.acm.org/ccs.cfm.
%% Please copy and paste the code instead of the example below.
%%
\begin{CCSXML}
<ccs2012>
<concept>
<concept_id>10002951.10003317.10003325.10003330</concept_id>
<concept_desc>Information systems~Query reformulation</concept_desc>
<concept_significance>500</concept_significance>
</concept>
<concept>
<concept_id>10002951.10003317.10003325.10003328</concept_id>
<concept_desc>Information systems~Query log analysis</concept_desc>
<concept_significance>500</concept_significance>
</concept>
<concept>
<concept_id>10002951.10003227.10003351.10003446</concept_id>
<concept_desc>Information systems~Data stream mining</concept_desc>
<concept_significance>500</concept_significance>
</concept>
</ccs2012>
\end{CCSXML}

\ccsdesc[500]{Information systems~Query reformulation}
\ccsdesc[500]{Information systems~Query log analysis}
\ccsdesc[500]{Information systems~Data stream mining}

%%
%% Keywords. The author(s) should pick words that accurately describe
%% the work being presented. Separate the keywords with commas.
\keywords{datasets, neural networks, gaze detection, text tagging}

%%
%% This command processes the author and affiliation and title
%% information and builds the first part of the formatted document.
\maketitle

\input{0abstract}
– project title
– team members: roles of each member
– description of the problem you try to address
– preliminary plan (milestones)
– paper list
\input{2method}
\input{3experiments}
\input{4related}
\input{5conclusion}

\bibliographystyle{ACM-Reference-Format}
\bibliography{bibliography}
%\appendix
\end{document}
\endinput

